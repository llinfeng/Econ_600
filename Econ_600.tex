%        File: Econ_600.tex
%     Created: Mon Aug 24 07:00 AM 2015 EDT
%
\documentclass[12pt]{article}
% Package for non-breaking words 
\usepackage[none]{hyphenat}
\usepackage[margin=1in]{geometry}

% For standard math
\renewcommand{\phi}{\varphi}
\renewcommand{\epsilon}{\varepsilon}
%Set up the margins
%\usepackage[left=.7in,right=.7in,top=.7in,bottom=.7in]{geometry}
\usepackage{amssymb,amsmath,amsthm,mathrsfs,verbatim}
\usepackage{extarrows}%% For extending the equal sign when there is   \xlongequal{<stuff>}
%\sout{} strike through!
\usepackage{marginnote}%\marginnote{}
\usepackage[colorlinks,linkcolor=blue]{hyperref}
%Create a header at the top of every page
%\usepackage{fancyhdr}
%\pagestyle{fancy}
\usepackage{pdfpages} %merge pdf with: \includepdfmerge{heine-borel_proof.pdf,-}
\usepackage{graphicx}%to input images \includegraphics[width=130mm]{Open_Open}
\usepackage{lipsum}
\usepackage{color}
%You can define commands for the things you use frequently.
\newcommand{\al}{\aleph}
\newcommand{\A}{\mathcal{A}}
\renewcommand{\c}{\mathfrak c}
\newcommand{\C}{{\mathbb C}}
\newcommand{\D}{\mathcal D}
\newcommand{\E}{\mathfrak E}
\renewcommand{\L}{{\mathcal L}}
\newcommand{\M}{\mathcal M}
\newcommand{\N}{{\mathbb N}}
\renewcommand{\P}{\mathcal {P}}
\newcommand{\Q}{{\mathbb Q}}
\newcommand{\R}{{\mathbb R}}
\newcommand{\X}{{\mathlarger{\mathcal X}}}
\newcommand{\Z}{{\mathbb Z}}
\newcommand{\To}{\Rightarrow}
\newcommand{\WLOG}{With out loss of generality}
\newcommand{\card}{\text{card }}

% The limits.
\newcommand{\limk}{\underset{k\to\infty}\lim}
\newcommand{\limn}{\underset{n\to\infty}\lim}
\newcommand{\liminfn}{\underset{n\to\infty}{\underline{\lim}}}
\newcommand{\liminfk}{\underset{k\to\infty}{\underline{\lim}}}
\newcommand{\liminfp}{\underset{p\to\infty}{\underline{\lim}}}
\newcommand{\liminfj}{\underset{j\to\infty}{\underline{\lim}}}
\newcommand{\limsupn}{\underset{n\to\infty}{\overline{\lim}}}
\newcommand{\limsupk}{\underset{k\to\infty}{\overline{\lim}}}
\newcommand{\limsupp}{\underset{p\to\infty}{\overline{\lim}}}
\newcommand{\limsupr}{\underset{r\to\infty}{\overline{\lim}}}
\newcommand{\sgn}{\text{sgn}}
% For L^p space
\newcommand{\Linfty}{{L^\infty(E)}}
\newcommand{\Lp}{{L^{p}(E)}}
\newcommand{\Lq}{{L^{q}(E)}}
\usepackage{mdframed}%needed for box like theorems.\newmdtheoremenv %In the box, the footnotes are more handy! ^.^
% Additional structures, enumerated through section number.
% Found as of 2015-06-30
% Fonts in the environments will be normal (standing straight up)
\theoremstyle{definition}
% Define all the theorem-based environments.
\newtheorem{theorem}{Theorem}[section]
\newtheorem{THM}{Theorem}
\newtheorem{acknowledgement}[theorem]{Acknowledgement}
\newtheorem{axiom}[theorem]{Axiom}
\newtheorem{algorithm}[theorem]{Algorithm}
\newtheorem{case}[theorem]{Case}
\newtheorem{claim}[theorem]{Claim}
\newtheorem{claimbox}[theorem]{Claim}
\newtheorem{conclusion}[theorem]{Conclusion}
\newtheorem{condition}[theorem]{Condition}
\newtheorem{conjecture}[theorem]{Conjecture}
\newtheorem{corollary}[theorem]{Corollary}
\newtheorem{criterion}[theorem]{Criterion}
\newtheorem{definition}[theorem]{Definition}
%\newtheorem{example}[theorem]{Example}
\newtheorem{exercise}[theorem]{Exercise}
\newtheorem*{exercise_nonum}{Exercise}
\newtheorem{fact}[theorem]{fact}
\newtheorem{idea}[theorem]{Idea}
\newtheorem{lemma}[theorem]{Lemma}
\newtheorem{notation}[theorem]{Notation}
\newtheorem{proposition}[theorem]{Proposition}
\newtheorem{question}[theorem]{Question}
\newtheorem{remark}[theorem]{Remark}
\newtheorem{remark_box}[theorem]{Remark}
\newtheorem{solution}[theorem]{Solution}
\newtheorem{summary}[theorem]{Summary}
\newmdtheoremenv{example}{Example}
\newmdtheoremenv{question_sqrt}{Question}
\newmdtheoremenv{typo}{Typo Correction}
% Found as of 2015-06-30
% Fonts will resume to be italic in for the environments thereafter.
\theoremstyle{plain}
\newtheorem{assumption}[theorem]{Assumption}
\newtheorem*{defn}{Definition}
\newtheorem*{rudin}{Rudin Says}
\newtheorem*{poblem}{\textcolor[rgb]{1.00,0.00,0.00}{Problem}}
\usepackage{mathrsfs} % enable people to use \mathscr{A}
\usepackage{color}
\usepackage{xcolor}
\newcommand{\hilight}[1]{\colorbox{yellow}{#1}}
\newcommand{\highlight}[1]{%
 \colorbox{yellow!50}{$\displaystyle#1$}}

\usepackage{bbm}
\usepackage{relsize} % For large symbles: \mathlarger{math_expression}
\usepackage{marvosym} %
\usepackage{enumerate}
%\usepackage{enumitem} % For using: \begin{itemize}[leftmargin=-.5in]
%\usepackage{comment}
% To make footnote numering by section.
\makeatletter
\@addtoreset{footnote}{section}
\makeatother
\usepackage{colonequals}

% Package for including footnotes to section-titles
\usepackage[stable]{footmisc}

% Highlighting:
\usepackage{xcolor}
\usepackage{newverbs}
\newverbcommand{\bverb}
{\begin{lrbox}{\verbbox}}
{\end{lrbox}\colorbox{blue!30}{\box\verbbox}}
\newverbcommand{\yverb}
{\begin{lrbox}{\verbbox}}
{\end{lrbox}\colorbox{yellow!50}{\box\verbbox}}
\newverbcommand{\gverb}
{\begin{lrbox}{\verbbox}}
{\end{lrbox}\colorbox{green!30}{\box\verbbox}}
\newverbcommand{\rverb}
{\begin{lrbox}{\verbbox}}
{\end{lrbox}\colorbox{red!30}{\box\verbbox}}
\newverbcommand{\grayverb}
{\begin{lrbox}{\verbbox}}
{\end{lrbox}\colorbox{gray!30}{\box\verbbox}}

% For := symbol \coloneqq
\usepackage{mathtools}


\title{Econ 600: taught by Prof. Shaowei Ke}
\author{Linfeng Li \\ llinfeng@umich.edu}



% Check-marks
\usepackage{pifont} % http://ctan.org/pkg/pifont
\newcommand{\cmark}{\ding{51}}%
\newcommand{\xmark}{\ding{55}}%

% Strike through
\usepackage[normalem]{ulem}
% Example: \sout{Hello World}

% Putting a box around paragraph
\usepackage{mdframed}
% \begin{mdframed}


\begin{document}
\maketitle

\section*{Disclaimer}
This is a personal note of mine. I will try to follow professor Ke's lecture as 
close as possible. However, neither is this an official lecture note, nor will 
Linfeng be responsible for any errors + typos. 
\underline{Nevertheless, corrections and suggestions are always welcomed.}

\smallskip

As this lecture note will be maintained on Github, PLEASE: 
\begin{itemize}
    \item Use the \href{https://github.com/llinfeng/Econ_600/issues}{``Issues''} feature on Github to post suggestions;
    \item Feel free to fork this repo and send me pull requests.
\end{itemize}
Paragraphs starting with ``Note that \ldots'' are most likely my personal 
reflections. Please be aware of this.

\section{Lecture 1: Logic, Sets and some Real Analysis\footnote{Relation, 
Function, Correspondence and Sequences in $\R$}}

\subsection{Logic}
\begin{definition}
    \textbf{Proposition} is a sentence that is either \textit{true} or 
    \textit{false}. It cannot be both true and false.
\end{definition}
Note: ``true'' and ``false'' may not necessarily be based on any (objective/subjective) 
factual basis. However, to give a concrete example, contextually correct 
propositions are usually employed.

\begin{definition}
    Logic Connectives: $\land$ and $\lor$. Let $P$ and $Q$ be propositions 
    \begin{itemize}
        \item Conjunction of $P$ and $Q$ is denoted as $P \land Q$;
        \item Disjunction of $P$ and $Q$ is denoted as $P \lor  Q$.
        \item Negation of $P$ is denoted as: $\neg P$.
    \end{itemize}
\end{definition}

\paragraph{Truth Table} is vaguely defined, with each row being a possible 
``state of the world''. On top of this, 

\begin{table}[htpl]
    \centering
    \begin{tabular}{c|c|c|c|c}
        $P$    & $Q$ & $P\land Q$ & $P\lor Q$ & $\neg P$ \\
        \hline
        1      & 1   & 1          & 1         & 0        \\
        1      & 0   & 0          & 1         & 0        \\
        0      & 1   & 0          & 1         & 0        \\
        0      & 0   & 0          & 0         & 1
    \end{tabular}
    \caption{Truth Table for logic connectives}
\end{table}

\begin{definition}[ Conditionals and Biconditionals] Let $P ,Q, R $ be 
    propositions, 
    \begin{enumerate}
        \item Conditional of $P$ and $Q$ is $P \implies Q$;
        \item Biconditional of $P$ and $Q$ is $P \iff Q$.
    \end{enumerate}

    
    \begin{table}[htpl]
        \centering
        \begin{tabular}{c|c|c|c}
            $P$    & $Q$ & $P \implies Q$ & $P \iff Q$ \\
            \hline
            1      & 1   & 1              & 1          \\
            1      & 0   & 0              & 0          \\
            0      & 1   & \yverb|1|              & 0          \\
            0      & 0   & \yverb|1|              & \bverb|1|
        \end{tabular}
        \caption{Truth Table for Conditionals and Biconditionals}
    \end{table}
    Note that, the two \yverb|1|'s are obtained for free. Conditional of $P$ and 
    $Q$ are trivially true if $P$ is false (thus the conditional is not entered, 
    thereby cannot be disproved?). \marginnote{$\Leftarrow$Check This.}

    Additionally, from \href{http://www.regentsprep.org/regents/math/geometry/gp1/ifthen.htm}
    {an external source} ($\leftarrow$ click me!):
    \begin{quote}
        % These font-modifiers are only valid within such environment.
        \small
        Conditionals are FALSE only when the first condition (if) is true and 
        the second condition (then) is false.  All other cases are TRUE.
    \end{quote}

\end{definition}

\begin{definition}
    Two propositions are \textbf{equivalent} if they have the same truth table, 
    denoted using ``$\equiv$''.
\end{definition}

\begin{example}
    Claim: that $P \implies Q$ and $\neg Q \implies \neg P$ are equivalent.

    \begin{proof}
        Refer to table \ref{table:equivalence_of_two_statements}: that by 
        definition, the truth table of the two conditionals are the same.
    \end{proof}

    Note, (it seems that)\footnote{Since ``truth table'' was not explicitly 
    defined.} truth tables are the same if the two ``column vectors'' denoting 
    the true/false status are the same.
\end{example}

\begin{table}[htpl]
    \caption{Truth Table: equivalence of $P \implies Q$ and $\neg Q \implies \neg P$}
    \bigskip
    \centering
    \begin{tabular}{c|c|c|c}
        $P$ & $Q$ & $P \implies Q$ & $\neg Q \implies \neg P$ \\
        \hline 
        1      & 1   & 1              & 1          \\
        1      & 0   & 0              & 0          \\
        0      & 1   & 1              & 1          \\
        0      & 0   & 1              & 1
    \end{tabular}
    \label{table:equivalence_of_two_statements}
\end{table}

\begin{definition}[Tautology]
    A proposition whose truth table consists only $1$'s is called 
    \textbf{tautology}.
\end{definition}

\begin{example}
    Claim: $Q \implies (P \implies Q)$ is a tautology.
    \begin{proof}
        Refer to Table  \ref{table:tautology_example}
    \end{proof}
\end{example}
\begin{table}[!htb]
    \caption{Truth Table: Tautology}
    \bigskip
    \centering
    \begin{tabular}{c|c|c|c}
        $P$    & $Q$ & $P\implies Q$ & $Q \implies (P \implies Q)$ \\
        \hline
        1      & 1   & 1             & 1 \\
        1      & 0   & 0             & 1  \\
        0      & 1   & 1             & 1 \\
        0      & 0   & 1             & 1
    \end{tabular}
    \label{table:tautology_example}
\end{table}


\begin{remark}
    We introduce the following 4 types of proof: 
    \begin{enumerate}
        \item Direct proof: to follow the direction of the statement.
            \begin{itemize}
                \item \textbf{Proposition}: For odd integers $x,y$, $x+y$ is an 
                    even integer.
            \end{itemize}
        \item Proof by contrapositive: (restate the proposition and prove the 
            easier direction).
            \begin{itemize}
                \item \textbf{Proposition}: If $n^2$ is an odd integer 
                    ($P$), then $n$ is an odd integer.
                    \begin{proof}
                        Prove instead that: ``if $n$ is an even integer, then 
                        $n^2$ is an even integer''.
                    \end{proof}
            \end{itemize}
        \item Proof by contradiction: (construct a structure that leads to 
            contradiction between derived conditions and given conditions.).
            \begin{itemize}
                \item That $\sqrt 2$ is rational number\footnote{The set of 
                    rational numbers is denoted as $Q$.}.
            \end{itemize}
        \item Proving a ``if and only if'' statement/proposition to be true: 
            either one of the following 4 are valid strategies: 
            \begin{enumerate}
                \item $P \implies Q$ and $Q \implies P$;
                \item $P \implies Q$ and $\neg P \implies \neg Q$;
                \item $\neg Q \implies \neg P$ and $Q \implies P$;
                \item $\neg Q \implies \neg P$ and $\neg P \implies \neg Q$.
            \end{enumerate}
    \end{enumerate}
\end{remark}

\subsection{Sets}
%\addtocounter{theorem}{6}
\begin{remark}[Russell's paradox]
    The barber is a man who shaves all those and only those who do not shave 
    themselves.

    In terms of set theory, let $R = \{ x : x \not \in x\}$, then:
    \[
        R \in R \iff R \not \in R
    \]
    which is very problematic.
\end{remark}

\begin{definition}[Sets]
    There are two definition of sets: 
    \begin{enumerate}
        \item (Enumerating all elements)

            A set is a collection of objects, e.g. $\{1,2,\ldots\}$ \footnote{a countably 
            infinite set.} or $\{1,2\}$ \footnote{a finite set.}.
        \item (Describing properties to be satisfied by elements in the set)

            If $A$ is a set of all objects that satisfies property $P$, then we 
            can write 
            \[
                A = \{ x : P(x)\}
            \]
            where the colon means ``such that'', and $P(x)$ means that $x$ 
            satisfies property $P$.
    \end{enumerate}
\end{definition}

Now, we can define the following \textbf{sets} using the two definitions of sets:
\begin{itemize}
    \item (Natural Number) $N = \{ 1, 2, \ldots\}$;
    \item (Integer) $Z = \{ x: x = n \text{ or } x = -n \text{ or } x = 0, 
        \text{ for some } n \in N\}$;
    \item (Rational number) $Q = \{ x : x = \frac{m}{n}, m,n \in Z\}.$
\end{itemize}

\begin{definition}
    [Set Equality]
    Two sets $A$ and $B$ are equal if they have the same elements. That is: 
    \[
        A = B \text{ if and only if } x \in A \iff x \in B, \forall x
    \]
    Note, that the notion $\forall x$ was used sloppily here. Without loss of 
    generality, it shall better be $\forall x \in A \bigcup B$.
\end{definition}

\begin{definition}
    [Set Containment]
    A set $A$ is contained in a set $B$, denoted by $A \subseteq B$, if $\forall 
    x\in A \implies x \in B$.

    As a consequence, $A = B$ if and only if $A \subseteq B$ and $B \subseteq 
    A$.

\end{definition}

\begin{definition}
    [Cardinality (finite case)]

    If a set $A$ has $n\in N$\footnote{Natural number.} distinct elements, then $n$ is the cardinality of $A$ 
    and we call $A$ a finite set. The \textbf{cardinality of $A$} is denoted by 
    $|A|$.
\end{definition}

\begin{definition}
    [Empty set $\emptyset$]
    The empty set is the set with no element. % denoted by $\emptyset$.
\end{definition}

\begin{definition}
    [Power set $2^A$]
    Let $A$ be a set. The \textbf{power set of $A$} is the collection of all 
    subsets of $A$. 

    Note that, $A$ is an arbitrary set. It could be finite, in which case $2^A$ 
    easy to envision; At the other extreme, it could be a uncountable set. 
    Nevertheless, the following equality shall hold: 

    \[
        |2 ^ A| = 2 ^{|A|}
    \]

    \begin{example}
        Let $A = \{ 1, 3\}$, then $2^A = \big\{ \emptyset, \{1\}, \{3\}, \{1,3\} 
        \big\}$. In terms of notation, note that $1$ is an element in $A$, thus 
        $1 \in A$; yet, $\{1\}$ is a subset of $A$, thus $\{1 \} \subset A$.
    \end{example}
\end{definition}


\begin{definition}
    [Operations on sets: $\bigcup$, $\bigcap$, $\setminus$ and $\cdot^c$.] Let 
    $A$ and $B$ be two sets:
    \verb| |

    \begin{itemize}
        \item Union: $A \bigcup B \coloneqq \{ x : x \in A \lor x \in B\}$;
        \item Intersection: $A \bigcap B \coloneqq \{ x : x \in A \land x \in 
            B\}$;
        \item $A$ and $B$ is disjoint if $A \bigcup B = \emptyset$;
        \item Difference of $A$ and $B$ is defined as: $A \setminus B \coloneqq 
            \{ x \in A \land x \not \in B\}$;
        \item Complements of $A$: $A^c \coloneqq \{ x : x \not \in A\}$.
    \end{itemize}
\end{definition}

\noindent Side note: \textbf{Index set} $I$ is a countable set. 
\[
    \underset{i \in I}{\bigcup} A_i 
    = 
    \{ x : x \in A_i \text{ for some } i \in I \}
\]



\begin{definition}
    [de Morgan's law]

    \[
        \left( \underset{i \in I } {\bigcup} A_i  \right)^c 
        =
        \underset{i \in I}{\bigcap} \left( A_i ^c \right)
        \text{ and }
        \left( \underset{i \in I } {\bigcap} A_i  \right)^c 
        =
        \underset{i \in I}{\bigcup} \left( A_i ^c \right)
    \]
\end{definition}

\begin{exercise}
    Prove that $\left( A \bigcup B \right)^ c = A^c \bigcap B^c$.
    \begin{proof}
        Prove mutual containment using element argument.
    \end{proof}
\end{exercise}


\noindent\rule{\textwidth}{1pt} % I am a line!
\begin{center}
    \vspace{-11pt} 
    Counters reset
\end{center}
    \vspace{-16pt}

\noindent\rule{\textwidth}{1pt} % I am a line!



\subsection{Relation, Function and Correspondence}
\setcounter{theorem}{0}
\begin{definition}
    [Ordered pair]
    For two sets $A$ and $B$, an ordered pair is $(a,b)$ such that $a \in A$ and $b 
    \in B$.
\end{definition}

\begin{definition}
    [$n$-taple]
    Let there be $n$ sets: $A_1, \ldots, A_n$, 
     an $n$-taple is $(a_1, \ldots, a_n)$ such that $a_i \in A_i$, $\forall i = 
    1,2,\ldots n$.
\end{definition}

\begin{definition}
    [Cartesian Product]

    Let $A_1, \ldots, A_n$ be non-empty sets. Cartesion product of $A_1, \ldots, 
    A_n$ is $A_1 \times \cdots \times A_n$, defined as: 
    \[
        \Pi_{i=1}^n A_i = \{ (a_1, \ldots, a_n): a_i \in A_i, \forall i  =1,  
        \ldots, n\}
    \]

\end{definition}

\begin{definition}
    [Relation]
    A relation from set $A$ to set $B$ is a subset of $A\times B$, denoted by 
    $R$. 
    \[
        a R b \iff (a,b) \in R
    \]
    A relation on $A$ is a subset of $A \times A$.

    \label{def:a_relation_on_A}
\end{definition}

\begin{definition}
    A relation $R \subseteq A \times A$ is said to be: 
    \begin{itemize}
        \item \emph{reflective} if $a R a$ $\forall a \in A$. (That is, $(a,a) \in R 
            $, $\forall a \in A$.);

        \item \emph{complete} if either $aRb$ or $b R a$, $\forall a, b \in A$;

        \item \emph{symmetric} if $\forall a, b \in A$, $aR b \implies b R a$;

        \item \emph{antisymmetric} if $\forall a, b \in A$, $a R b $ and $ b R a 
            \implies a = b$.

        \item \emph{transitive} if $\forall a,b,c \in A$ s.t. $aRb$ and $bRc$, 
            $aRc$ (is implied).
    \end{itemize}
\end{definition}

\begin{table}[htpl]
    \caption{Property of common relations}
    \bigskip
    \centering
    \begin{tabular}{c|c|c|c|c|c}
                        & $<$    & $\le$  & $\in$  & $\subseteq $ & $\succeq$ \\
          \hline
          reflective    & \xmark & \cmark & \xmark & \cmark       & \cmark    \\
          complete      & \xmark & \cmark & \xmark & \xmark       & \cmark    \\
          symmetric     & \xmark & \xmark & \xmark & \xmark       & \xmark    \\
          antisymmetric & \cmark & \cmark & \cmark & \cmark       & \xmark    \\
          transitive    & \cmark & \cmark & \xmark & \cmark       & \cmark
    \end{tabular}
\end{table}


Note that, $<$ and $\le$ are defined on $\R$; $\in$ and $\subseteq$ are 
defined on sets; $\succeq$ is preference relation that represents ``weakly 
prefer''.

Also note that, completeness implies reflectiveness.

\begin{definition}
    [Equivilance relation]

    An \textbf{equivalence} on set $A$ is a relation $E$ that is 
    \emph{reflective, symmetric and transitive}. It is denoted as $\sim$.

    For any $a \in A$, the \textbf{equivalence class} of $a$ with respect to 
    $\sim$ is defined to be the set 
    \[E_\sim(a) = \{ b \in A, b\sim a\}\]
\end{definition}

Remark: by construction in Definition \ref{def:a_relation_on_A},  equivalence 
($\sim$) is defined as ``a relation on $A$'', which is thereby defined in the 
Cartesian space.

\begin{definition}
    [Fucntion: defined using Relation from $A$ to $B$]
    A function from set $A$ to set $B$ is a relation $f$ from $A$ to $B$ such 
    that: 
    \begin{enumerate}[(i)]
        \item $\forall a \in A$, $\exists b \in B$ such that $(a,b) \in f$, i.e. 
            $a f b$
        \item $\forall a \in A$, if $(a,b) \in f$ and $(a,c) \in f$ for some $b, 
            c \in B$, then $b = c$.
    \end{enumerate}

    Note that, alternatively, the two conditions could be written in short as: 
    \begin{enumerate}[(iii)]
        \item $\forall a \in A$, $\exists ! b \in B$ such that $(a,b) \in f$, i.e. 
            $a f b$
    \end{enumerate}

\end{definition}
\paragraph{Convention for $f$:} If $(a,b) \in f$, we write $f(a) = b$. And, $f$ 
could be interpreted as a ``mapping'': ``$f: A\to B$''.

\begin{definition}[Domain and Rnage]
    If $f$ is a function from $A$ to $B$, then $A$ is called the 
    \textbf{domain } of $f$ and $B$ is the \textbf{codomain} of $f$. The 
    \textbf{range} of $f$ is the set:
    \[
        Ran (f) = \{ b \in B: \exists a \in A 
        \text{ s.t. } f(a) = b\}.
    \]
\end{definition}

\begin{definition}
    [Propoteries of functions] Let $f$ be a function, then: 
    \begin{enumerate}[(i)]
        \item $f$ is \textbf{surjective} if $Ran(f) = B$; \marginnote{onto}
        \item $f$ is \textbf{injective} if $a_1 \not = a_2 \in A \implies f(a_1) 
            \not = f(a_2)$; \marginnote{1-to-1}
        \item $f$ is bijective if $f$ is subjective and injective.
    \end{enumerate}
\end{definition}

\noindent Side note: a \textit{indicator function} is defined as following: for 
$A$ being a set and $S\subseteq A$, 
\[
    \mathcal X_S(a) = 
    \begin{cases}
        1 & \text{ if } a \in S \\
        0 & \text{ otherwise }
    \end{cases}
\]

\begin{definition}
    [Image and Preimage]

    For $f: A \to B$ and $C\subseteq A$, \underline{ the \textbf{image} of $C$ under 
    $f$ is }
    \[
        f(C) = 
        \{ 
            b \in B : \exists a \in C \text{ s.t. } f(a) = b
        \}
    \]
    \underline{The \textbf{preimage} of $D\subseteq B$} is 
    \[
        f^{-1}(D) = 
        \{
            a \in A : f(a) \in D
        \}
    \]

\end{definition}

\begin{exercise_nonum}
    Prove that 
    \begin{enumerate}
        \item $f^{-1}(f(A)) = A$, and
        \item $f(f^{-1}(B)) = B$ if and only if $f$ is subjective.
    \end{enumerate}
\end{exercise_nonum}

\begin{proposition}
    Given $f: A \to B$, then $f^{-1}: B \to A$ is a function if and only if $f$ 
    is bijective.
\end{proposition}

\begin{definition}
    [Sequence]

    A sequence is a function $f: N \to A$, denoted by $\{a_1, a_2, \ldots\} = \{ 
    a_i \}_{i=1}^\infty $\footnote{This is an ordered set.}
    i.e. the set of all sequence is the following set: 
    \[
        A^\infty = A \times A \times \cdots
    \]
\end{definition}

\begin{definition}
    [Cardinality, for (infinite) sequences]
    Two sets $A, B$ have the same cardinality if $\exists$ a bijective function 
    $f : A\to B$. 

    Then, $|A|\ge |B|$ if there exists an injective function $f: B \to A$.
    (Example: $|Z|\ge |N|$ by using identify mapping from $N$ to $Z$; $|N|\ge 
    |Z|$ by enumerating elements in $Z$ using $N$. Thus, $|Z| = |N|$.) 
    Eventually, we have: 
    \[
        |\R^2| = |\R| > |Q| = |Z| = |N|
    \]

\end{definition}

\begin{definition}
    [Correspondence]
    $T: A \rightrightarrows B$ is a correspondence such that $T: A \to 2^A
    \setminus \emptyset$.
\end{definition}

\subsection{Sequences}
\setcounter{theorem}{0}
\begin{definition}
    [Sequence in $\R$]
    A sequence of real number is a function $a : N \to \R$ s.t. $a(i) = a_i$ is 
    the $i$-th component of the sequence $\{a_j\}_{j=1}^\infty$.
\end{definition}
\begin{definition}
    [Increasing sequence]
    A real sequence is increasing if $a_{n+1} \ge a_n$ $\forall n \in N$.
\end{definition}
\begin{definition}
    [Bounded and Bounded (from) above/below]
    A real sequence is
    
    \begin{itemize}
        \item \textbf{bounded above} if $\exists \bar m \in \R$ s.t. $a_n \le 
            \bar m$ $\forall n \in N$.

        \item \textbf{bounded below} if $\exists \underline m \in \R$ s.t. $a_n 
            \ge \underline m$ $\forall n \in N$.
        \item \textbf{bounded} if it is bounded above and bounded below.
    \end{itemize}
\end{definition}

\begin{definition}
    [Least upper bound]
    $a \in \R$ is the least upper bound of a sequence $\{a_n\}$ if 
    \begin{enumerate}[(i)]
        \item $a$ is an upper bound;
        \item $a$ is the smallest upper bound, i.e. $\not \exists b \in \R$ s.t. 
            $b < a$ and $b$ is a upper bound of $\{a_n\}$.
    \end{enumerate}
\end{definition}

\begin{axiom}
    [Axiom of Real Number: completeness axiom]
    If $S$ is a nonempty set of real numbers that is bounded above, then there 
    exists a least upper bound \sout{that is also a real number}.

    Note, that, claiming that the upper bound is in $\R$ is redundant.

\end{axiom}

\begin{definition}
    [Convergence sequences]
    A real sequence $\{a_n\}$ converges to the limit $a\in \R$ if $\forall 
    \varepsilon > 0$, $\exists N$ s.t. $\forall n \ge N$
    \[
        |a_n - a| < \varepsilon
    \]
    We write $\underset{n\to\infty} \lim a_n = a$ or $a_n \to a$.
\end{definition}

\begin{itemize}
    \item If a sequence does not converge, then it diverges. (To $+\infty$ or 
        $-\infty$.)
\end{itemize}

\begin{THM}
    A monotone bounded sequence converges.
    \begin{proof}
        Discuss two cases where 1) $\{a_n\}$ is an increasing sequence, and 2) $\{a_n\}$ 
        is a decreasing sequence. Then, proof is completed through using either 
        least upper bound (for increasing sequence) or largest lower bound (for 
        decreasing sequence).
    \end{proof}
%    \begin{proof}
%        Let $a = \sup a_n$ (the least upper bound of $\{a_n\}$), we want to show 
%        that: $\forall \varepsilon > 0$, $\exists N $ s.t. $\forall n > N$, 
%        $|a_n -a | < \varepsilon$
%    \end{proof}
\end{THM}


\section{Lecture 2: convergence and more}
\setcounter{theorem}{0}

\begin{definition}
    A set $S \subset X$ is a linearly ordered set if there is a relation 
    ``$\le$'' on $X$ s.t.

    \begin{center}
        $\le$ is complete, transitive and antisymmetric.
    \end{center}

    Note that, given the linear ordering, we can define $<$ accordingly. 
    (For arbitrary $a,b \in X$ and $a \le b$, then we say $a<b$ if $a \le b$ and 
    $a \not = b$.)
\end{definition}

\begin{definition}
    [Boundedness for an arbitrary set.]

    Let $X$ be a linearly ordered set and $S \subset X$, then $a \in X$ is 
    the \textbf{supremum} (or \textit{least upper bound) }of $X$ if: 
    \begin{enumerate}
        \item $a$ itself is an upper bound of $S$, i.e. 
        \item for $b \in X$, $b < a$, then $b$ is not an upper bound of $S$.
    \end{enumerate}

    \textbf{Corolloary: } For $a = \sup X$, $\forall \varepsilon > 0$, there 
    exists $x \in S$ s.t. $x > a - \varepsilon$.
\end{definition}

\begin{axiom}
    [Completeness Axiom] 
    If $S$ is a nonempty set of real numbers that is bounded above, then there 
    exists a least upper bound.
\end{axiom}

\begin{definition}
    [Sequence in $\R$]
    A sequence of real number is a function $a : N \to \R$ s.t. $a(i) = a_i$ is 
    the $i$-th component of the sequence $\{a_j\}_{j=1}^\infty$.
\end{definition}

%Question on this. Why we shall need this?
\begin{remark}
    $\{a_n\}$ is bounded if $a(N)$ is bounded.

    Note, here $N$ is the set of all natural numbers $\{1,2,\ldots,\}$. Thus, 
    we hereby define the boundedness of a sequence using the our previous 
    definition of set-boundedness.
\end{remark}

\begin{lemma}
    \label{lemma:1_B_W_Thm}
    A monotone bounded sequence converges.
\end{lemma}

\begin{definition}
    [Subsequence]

    A subsequence $\{a_{n_i}\}$ of $\{a_n\}$ is a sequence s.t. $1 \le n_1 \le 
    n_2 \le \ldots$. That is: 

    $ \exists $  conversion function $\Phi : N \to N$ s.t. $n_ i = \Phi(i)$ and 
    $\Phi(i) < \Phi(j)$ whenever $i < j$. We can also write: $a_{n_i} = 
    a_{\Phi(i)}$.
\end{definition}

\begin{lemma}
    \label{lemma:2_B_W_Thm}
    Every sequence of $\R$ has a monotone subsequence.

    \begin{proof}
        Proof by doodling: try to construct a decreasing sequence first, if 
        failed (cannot identify infinitely many of elements as candidate of the 
        sequence), construct an increasing one.

        Formally: let $S = \{ i : \text{ if } j > i, \text{ then } a _j < a_i 
        \}$. 
        \begin{itemize}
            \item if $|S| = |N|$ (countably infinite)
                \footnote{Writing $|S| = \infty$ is not rigorous enough, since 
                uncountably infinite could also be denoted similarly.}, we have 
                found a monotone (decreasing) sequence.

            \item If $|S| < \infty$, let $\max S = N$, then by construction, 
                $\exists n_1$ s.t. $a_{n_1} \ge a_{N+1}$. Since $n_1 \notin X$, 
                there exists $n_2 > n_1$ s.t. $a_{n_2} \ge a_{n_1} \ge a_N$.

                We can construct an increasing sequence in this fashion.
        \end{itemize}
    \end{proof}
\end{lemma}

\begin{theorem}
    [Bolzano-Weierstrass Theorem] %BW theorem
    A bounded sequence of $\R$ has a convergent subsequence.
    \begin{proof}
        By Lemma \ref{lemma:2_B_W_Thm}, such bounded sequence of $\R$ has a 
        monotone subsequence, which inebriates the boundedness property.
        
        Thus, by Lemma \ref{lemma:1_B_W_Thm}, such bounded monotone sequence 
        converges.
    \end{proof}
\end{theorem}

\begin{remark}
    [Properties of Limites]
    For $a_n \to a$ and $b_n \to b$ (two convergent sequences):
    \begin{enumerate}[(i)]
        \item $c \cdot a_n \to c\cdot a$, for $c \in \R$;
        \item $a_n + b_n \to a + b$
        \item $a_n \cdot b_n \to a\cdot b$
        \item $\frac{a_n}{b_n} \to \frac{a}{b}$ s.t. $b \not = 0$ and $b_n \not= 
            0$ $\forall n$.
        \item $\forall n \in N$, if  $c \le a_n$, then $c \le a$. (Note that we have 
            defined only one linear ordering $\le$.)

            However, $a_n > c$ does not imply $a > c$. (e.g.: 
            $\frac{1}{n} > 0, \forall n$, yet $\frac{1}{n}\to 0 = 0$.)

        \item $\forall n$, if $b_n \le a_n$, then $b\le a$.
    \end{enumerate}    
\end{remark}

\begin{definition}
    [Cauchy sequence]

    $\{a_n\}$ is a Cauchy sequence if $\forall \varepsilon> 0$, $\exists N$ s.t. 
    $\forall m , n \ge N$, $|a_m-a_n| < \varepsilon$.
\end{definition}

Note that, since the definition of convergent sequence relies on knowing the 
limit $a$, when such limit is not attainable, Cauchy becomes handy.

\begin{theorem}
    Every convergent sequence is Cauchy.
    \begin{proof}
        Given $\{a _n\} \to a$, thus $\forall \frac{\varepsilon}{2} > 
        0$ $\exists N$ s.t. $|a_n - a| < \frac{\varepsilon}{2}$, $\forall n > 
        N$.

        Now, for any $m, n \ge N$, we have: 
        \[
            \aligned
            |a_m - a_n| & = | a_m - a + a - a_n|\\
            & \le |a_m - a| + |a_n - a| < \varepsilon
            \endaligned
        \]
    \end{proof}
\end{theorem}

\paragraph{Example}: 
Prove that $a_{n+1} = \frac{a_n + 2a_{n-1}}{3}$ converges for $a_1 = 0$, $a_2 = 1$.

\begin{proof}
    \begin{enumerate}[Step 1]
        \item First observe that: $a_n$ is an average of two real numbers that are in $[0,1]$. Thus, 
            $a_n \in [0,1]$.

        \item Also observe that by rearranging the terms in the equality, we 
            have: 
            \[
                \frac{a_{n+1} - a_n}{a_n - a_{n-1}} = - \frac{2}{3}
            \]
    \end{enumerate}
    At this point, we check definition of Cauchy sequence by showing that: for 
    arbitrary $\epsilon$, we can find a $N$ such that $|a_m - a_n| < 
    \varepsilon$. Deriving the functional form of $|a_m  -a_n|$ suffices. (We 
    can then use this functional form to find a proper $N$.)

    Without loss of generality, let $m > n$, then:
    \[
        \aligned
        |a_m - a_n|& = |a_n - a_{n+1} + a_{n+1} - \cdots - a_m|\\
        & \le |a_n - a_{n+1}| + |a_{n+1} - a_{n+2} | + \cdots + |a_{m-1} - a_m| 
        \\
        & \le \left( \frac{2}{3} \right)^{n-1} + \left( \frac{2}{3} \right)^n + 
        \cdots + \left( \frac{2}{3} \right)^{m-2} \\
        & = \frac{\left( \frac{2}{3} \right)^{n-1}\left( 1 - \left( 
                    \frac{2}{3} 
        \right)^{m-n+2} \right)}{1 -  \frac{2}{3}} \\
        & = O(\left( \frac{2}{3} \right)^n)
        \endaligned
    \]
    By now, we can easily demonstrates that the definition of Cauchy sequence 
    could be satisfied by choosing a proper $N$ for any given 
    $\varepsilon$.
\end{proof}

\begin{theorem}
    Every Cauchy sequence is bounded.
    \begin{proof} Let $\{a_n\}$ be an arbitrary Cauchy sequence. Then, for for 
        arbitrary $\varepsilon>0$, we know that $\exists N_\varepsilon > 0$ such 
        that $\forall m,n > N$, $|a_m - a_n | < \varepsilon$.

        Now, to construct an upper bound for $\{a_n\}$, without loss of 
        generality, let $\varepsilon = 1$. Then, we know that there exists $N_1 
        > 0$ such that $\forall n,m > N_1$, $|a_n - a_m| < 1$. Then, let $M_1$ 
        denote the bound (either upper or lower). Then, in absolute value, we can 
        define it to be: 
        \[
            |M_1| = \max \{|a_1|, \ldots, |a_{N_{1}}|, |a_{N_1+1}| +  
            1\}
        \]
        Through more careful, yet unnecessary, discussions, we can derive the 
        exact bound using the absolute value $|M_1|$.

        Note that, the bound we found above is only \textit{one of the upper 
        bound}. It is not necessarily the $\sup$ nor $\inf$.
    \end{proof}
\end{theorem}

\begin{theorem}
    Every Cauchy sequence \yverb|in |$\highlight{\R}$ \footnote{Note that, for 
        $\{\frac{1}{n}\}$ defined on $(0,1]$, it does not converge in this space 
        since $0\not\in (0,1]$.}
        converges.
\end{theorem}

\begin{remark}
    [Useful limits] Limits of sequences as $n \to \infty$:

    Refer to page 57 of \cite{rudin1976principles} Theorem 3.20.
\end{remark}

\begin{definition}
    [limsup, liminf]
    Let $\{a_n\}$ be a sequence in $\R$, we say: 
    \[
        \lim\sup \{a_n\} = a
    \]
    if $\sup S = a$, where $S = \{ b \in \R : \exists \text{ subsequence } 
    \{a_{n_i}\} s.t. a_{n_i} \to b\}$.
\end{definition}

\paragraph{Exercise: equivalent definition of limsup} Prove that $\lim\sup a_n = 
a$ if and only if: 
\begin{enumerate}[(i)]
    \item $\forall \varepsilon > 0$, $\exists N > 0$ s.t. $a_n < a + 
        \varepsilon$, $\forall n > N$;
    \item $\forall \varepsilon > 0$, $\forall n \in N$, $\exists k > n$ s.t. 
        $a_k > a - \varepsilon$.
\end{enumerate}
Note that, (i) specified a property for subsequence; and (ii) is merely about 
the existence of one element in the sequence, to be found for all $(\varepsilon, 
n) \in \R_{++} \times N$.

\begin{proof} The iff statement will be established in the following three steps:
    \begin{itemize}
        \item Prove that $\lim\sup a_n = a$ implies (i).
    \end{itemize}
    WTS: $\forall \varepsilon > 0$, $\exists N > 0$ s.t. $a_n < a + 
        \varepsilon$, $\forall n > N$;

        First, suppose that $a = +\infty$, that is $\{a_n\}$ is not bounded from 
        above. Then we are done.

        Then, suppose that $\{a_n\}$ is bounded from above. We now prove by 
        contradiction. Suppose that $\exists \varepsilon > 0$ s.t. no such $N 
        \in \N$ exists. Then, we know that $1$ cannot serve the role of $N$. 
        So, for some $n_1 > 1$, 
        \[
            a_{n_1} \ge a + \varepsilon
        \]
        Still, $n_1 + 1$ cannot serve the role of $N$, then for some $n_2 > n_1 
        + 1$, 
        $$a_{n_2} \ge a + \varepsilon$$
        By induction, we can construct a subsequence that is bounded from below by 
        $a + \varepsilon$. Note that, the original sequence is bounded from 
        above, by Bolzano-Weierstrass Theorem, we know that a bounded sequence 
        converges. However, the limit of such subsequence shall be larger than $a$, 
        contradicting $\lim\sup a_n = a$.

        Thus what we assumed is wrong. We thereby proved the original claim in 
        (ii).


    \begin{itemize}
        \item Prove that $\lim\sup a_n = a$ implies (ii).
    \end{itemize}
    WTS: Given that $\lim\sup a_n = a$,
    $\forall \varepsilon > 0$, $\forall n \in N$, $\exists k > n$ s.t. 
        $a_k > a - \varepsilon$.

    Now, for arbitrary $\varepsilon > 0$, by definition of limsup, we know that 
    $\exists 
    a' \in (a - \frac{\varepsilon}{2}, a)$ s.t. $\exists \{a_{n_j}\}$ (a 
    subsequence of $\{a_n\}$) s.t. $a_{n_j} \to a'$.

    For this convergent subsequence per se, given the arbitrary $\varepsilon$ we 
    have specified in the very beginning, we know that $\exists J > 0$ s.t. 
    \[
        |a_{n_j} - a'| < \frac{\varepsilon}{2}, \qquad 
        \text{ for all } j > J
    \]

    Now, for arbitrary $n \in N$, we can always find a $k = n_i$ with $i> J$, 
    such that $a_k = a_{n_i}$ is within $\frac{\varepsilon}{2}$ distance away 
    from $a'$. Combining this fact with the construction that $a' \in (a - 
    \frac{\varepsilon}{2}, a)$, it is clear the $a_k$ we found specifically for 
    $\varepsilon$ and $n\in N$ satisfies: $a_k > a - \varepsilon$.
    


%    \begin{mdframed}
%       These are due to incorrect reading of (ii) in the proposition.
%
%        Now, for elements in $\{a_n\}$, $\forall k > n_J$: if $a_k \in \{a_{n_j}\}$, 
%        then by construction we have: $a_k > a - \varepsilon$; if $a_k \notin 
%        \{a_{n_j}\}$, then \textbf{it is not necessary that } $a_k > a - 
%        \varepsilon$. These points may well diverge to $-\infty$, or pursue a limit 
%        of $a' < a$ as a subsequence.
%
%    \end{mdframed}

    \begin{itemize}
        \item Prove that (i) and (ii) implies that $\lim\sup a_n = a$.
    \end{itemize}
    To prove that $\lim\sup a_n = a$, we first show that $a$ is the limit of a 
    subsequence of $\{a_n\}$; then we show that $\not\exists a' > a$ s.t. $a'$ 
    is the limit of a subsequence of $\{a_n\}$.

    Firstly, by (i) and (ii), for arbitrary $\varepsilon>  0$, we can find a 
    subsequence $\{a_{n_j}\}$ with certain $N\in \N$ 
    such that $a - \varepsilon < a_{n_j} < a + \varepsilon$, $\forall n_j > N$. 
    (Step 1: by (i), we can find a $N^\varepsilon$ for arbitrary $\varepsilon > 
           0$, so that: $a_n < a + \varepsilon$ $\forall n > N^\varepsilon$; Step 2, for the $\varepsilon$ 
           and all $\tilde n \ge N^\varepsilon$, we can find a $a_{k_{\tilde n}}$ 
           s.t.  $a - \varepsilon < a_{k_{\tilde n}}$. Thus, we have composed a 
       subsequence $\{a_{k_{\tilde n}}\}$.)

    Then, suppose $\exists a' > a$ as the limsup, then $\forall \varepsilon > 0$ 
    $\exists N' $ s.t. $\forall n' > N'$, $|a_{n'} - a'| < \varepsilon$. 
    However, (i) is violated when $\varepsilon < \frac{a' - a}{2}$: suppose that 
    $a_{n_k} \to a'$. Then, $\exists N' > 0$ s.t. $\forall k > N'$, $|a_{n_k} - a'| < 
    \varepsilon$. Given that $\varepsilon < \frac{a'-a}{2}$, there does not 
    exist a $N$ that may satisfy (i). (The ``$\forall n > N$'' statement is 
    violated due to the subsequence that converges to $a'$.)



    \bibliographystyle{plainnat}
    \bibliography{ref}

\end{proof}

\end{document}
